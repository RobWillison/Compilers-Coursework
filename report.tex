\documentclass{article}
\usepackage{graphicx}
\usepackage{listings}
\usepackage{tikz}
\usetikzlibrary{positioning,fit,calc}
\tikzset{block/.style={draw,thick,text width=2cm,minimum height=1cm,align=center},
         line/.style={-latex}
}
\begin{document}

\title{CM30171}
\author{Rob Willison}

\maketitle
\tableofcontents

\section{Introduction}
This report is an explanation of the design decisions undertaked while writing
a --C compiler and interpreter.
\newpage
\section{Code Interpretation}

\section{Intermediate Code Generation}

\subsection{TAC Design}

The three address code used in the compiler was designed to be abstract enough as
too keep any machine dependent decisions out of the TAC stage. Many of the instructions
are obvious such as store and mathematic operations, examples below, and they won't be explained in
great detail.

\begin{lstlisting}
r2 := 1
r3 := r1 / r2
\end{lstlisting}

One thing to note about the store instruction is that in the event that the store
operand is defined in the scope above the current scope like x is in the following example.

\begin{lstlisting}
int main()
{
  int x = 4;
  int test()
  {
    return x;
  }

  return test();
}
\end{lstlisting}

In that scenario the store command in the test function will have an extra piece
of information saying that its defined in the scope one level above, so the store command will
look like the following.

\begin{lstlisting}
DEFINED IN 1 r2 := x
\end{lstlisting}

This information is also included when a closure is called from a narrower scope,
the TAC would be as follows.

\begin{lstlisting}
CALL _1 FROM SCOPE 1
\end{lstlisting}

For Functions a label instruction denotes the start and a end label for the end.
After the start label the new activation frame instruction which tells
the compiler to allocate space for a new activation frame with a given number
of arguments, locals and tempories. Before the end there is a return instruction
which contains the register with the value to return. Below is an example function
in TAC.

\begin{lstlisting}
_1:
NEW FRAME 0 arg 0 loc 1 temp
DEFINED IN 1 r2 := x
RETURN r2
FUNCTION END
\end{lstlisting}

There is one type of control sequence in the language, the if else, the way this
is described in TAC is using a sequence of if instructions and labels denoting the
various bodies if the if and else parts. An example in --C and the corresponding TAC
are below.

\begin{lstlisting}
if (1 > 4) {
  return 4;
} else if (2 > 1) {
  return 3;
}
\end{lstlisting}

\begin{lstlisting}
r1 := 1
r2 := 4
r3 := 1 > 4
IF NOT r3 GOTO 1
r4 := 4
RETURN 4
GOTO 2
LABEL 1: r5 := 2
r6 := 1
r7 := 2 > 1
IF NOT r7 GOTO 3
r8 := 3
RETURN 3
LABEL 3: LABEL 2:
\end{lstlisting}

Finally there is one loop type in the language, the while loop, this is translated
into a goto, an if and a label at the top and bottom of the loop body. the while
condition is placed after the body. n example in --C and the corresponding TAC
are below.

\begin{lstlisting}
int x = 0;
while (x < 5)
{
  x = x + 1;
}
\end{lstlisting}

\begin{lstlisting}
r1 := 0
x := r1
GOTO 1
LABEL 2: r2 := x
r3 := 1
r4 := r2 + r3
x := r4
LABEL 1: r5 := x
r6 := 5
r7 := r5 < r6
IF r7 GOTO 2
\end{lstlisting}

\subsection{TAC Generation}

All the TAC complilation is done in the tac\_compliler.c file.
In order to generate the TAC from source code a tree walk is performed over the
parse tree at each node depending on the type a set of TAC instructions are created
and added to the current TAC block. If a new function is found a new TAC block is
created before that part of the tree is parsed, also if a goto or label TAC instruction
are created new blocks are created. When a leaf is reached a store instruction is
created for the value at the leaf.\\~\\

In order to keep track of the location of any local variables in the code and enviroment
is created to store the association between token and register location. The enviroment
is made up of frames each frame corresponds to a scope in the program and has
a linked list of all locals in that scope. The frames also contain a linked list of
the functions defined, these are a pair of token and label assigned to the function.
For example the environment created while walking the following piece of code is shown below.

\begin{lstlisting}
int main()
{
  int x = 4;
  int test()
  {
    int y = 5;
    return x * y;
  }

  return test();
}
\end{lstlisting}

\begin{tikzpicture}
  \node[block] (a) {Frame 1};
  \node[block,right=of a] (b) {Frame 2};
  \node[block] (c) at [yshift=-2cm] {x - r1};
  \node[block] (d) at [yshift=-4cm] {test - \_0};
  \node[block, right=of a] (e) at [yshift=-2cm, xshift=1.2cm] {y - r2};
  \draw[line] (a)-- (b);
  \draw[line] (a)-- (c);
  \draw[line] (c)-- (d);
  \draw[line] (b)-- (e);
\end{tikzpicture}

\subsection{TAC Optimisation}

\section{Machine Code Generation}

\end{document}